\documentclass[12pt,a4paper]{article}
\usepackage[utf8]{inputenc}
\usepackage[french]{babel}
\usepackage[T1]{fontenc}
\usepackage{geometry}
\usepackage{hyperref}
\usepackage{enumitem}
\usepackage{xcolor}

\geometry{margin=2.5cm}

\title{\textbf{TP sur les données structurées - CORRIGÉ}\\
\large Partie 1 : Tableur et tris\\
Films ayant réalisé plus d'un million d'entrées}
\author{}
\date{}

\begin{document}

\maketitle

\section{Analyse sommaire du fichier}

\textbf{Question 1 :} Quel est le film ayant fait le plus d'entrées en 2024 ?

\textcolor{blue}{\textbf{Réponse :}} Il s'agit du film \textbf{\og{}UN P'TIT TRUC EN PLUS\fg{}} avec 10,72 millions d'entrées (10~716~771 entrées exactement).

\vspace{0.5cm}
\textbf{Question 2 :} Quel est le film français ayant fait le plus d'entrées en 2023 ?

\textcolor{blue}{\textbf{Réponse :}} En 2023, le film français ayant réalisé le plus d'entrées est \textbf{\og{}ASTÉRIX ET OBÉLIX ET L'EMPIRE DU MILIEU\fg{}} avec 4,61 millions d'entrées (4~605~168 entrées).

\vspace{0.5cm}
\textbf{Question 3 :} Quelle est la nationalité du film ayant fait le plus d'entrées en 2023 ?

\textcolor{blue}{\textbf{Réponse :}} Le film ayant réalisé le plus d'entrées en 2023 est \og{}SUPER MARIO BROS. LE FILM\fg{} qui est un film \textbf{américain (US)} avec 7,25 millions d'entrées.

\section{Utilisation des filtres}

\textbf{Question 4 :} Quels sont les trois films français ayant fait le plus d'entrées en 2024 ?

\textcolor{blue}{\textbf{Réponse :}} Les trois films français ayant réalisé le plus d'entrées en 2024 sont (dans l'ordre) :
\begin{enumerate}
\item \textbf{UN P'TIT TRUC EN PLUS} - 10,72 millions d'entrées
\item \textbf{LE COMTE DE MONTE-CRISTO} - 9,39 millions d'entrées
\item \textbf{L'AMOUR OUF} - 4,83 millions d'entrées
\end{enumerate}

\textit{Démarche :} Onglet 2024, appliquer un AutoFiltre sur la ligne 7, filtrer la nationalité en ne gardant que \og{}FRANCE\fg{}, observer les premières lignes.

\vspace{0.5cm}
\textbf{Question 5 :} Combien de films français ont réalisé plus d'un million d'entrées en 2024 ?

\textcolor{blue}{\textbf{Réponse :}} \textbf{9 films français} ont réalisé plus d'un million d'entrées en 2024.

\textit{Démarche :} Avec le filtre français actif, utiliser \texttt{=SOUS.TOTAL(3;B8:B43)} pour compter le nombre de lignes filtrées.

\vspace{0.5cm}
\textbf{Question 6 :} Quel est le total d'entrées réalisées par ces films français en 2024 (en millions) ?

\textcolor{blue}{\textbf{Réponse :}} Le total des entrées des films français en 2024 est d'environ \textbf{37,39 millions d'entrées}.

\textit{Démarche :} Avec le filtre français actif, utiliser \texttt{=SOUS.TOTAL(9;E8:E43)} pour sommer les entrées.

\vspace{0.3cm}
\textit{Méthode alternative :} Sélectionner les cellules E8:E43 filtrées et regarder la barre d'état en bas de l'écran qui affiche automatiquement la somme.

\vspace{0.5cm}
\textbf{Question 7 :} Parmi les films affichés en 2024, certains sont sortis en 2023 (comme \textit{Wonka} sorti le 13 décembre 2023). Expliquez pourquoi ces films apparaissent dans la liste des films les plus vus en 2024.

\textcolor{blue}{\textbf{Réponse :}} Ces films sont sortis en fin d'année 2023 (décembre). Par conséquent, ils ont réalisé la majorité de leurs entrées en 2024, après leur sortie. C'est pourquoi ils apparaissent dans la liste des films les plus vus en 2024, même s'ils sont officiellement sortis en 2023.

\section{Exercice : Analyse des films de 2023 et 2024}

\subsection*{Questions sur l'année 2024}

\textbf{Question 8 :} Combien de films américains (ETATS UNIS) sortis en 2024 (et non en 2023) ont réalisé plus d'un million d'entrées ? Donnez également le titre des trois premiers.

\textcolor{blue}{\textbf{Réponse :}} \textbf{13 films américains} sortis en 2024 ont réalisé plus d'un million d'entrées.

Les trois premiers sont :
\begin{enumerate}
\item VICE-VERSA 2 - 8,29 millions d'entrées
\item VAIANA 2 - 6,68 millions d'entrées
\item MOI, MOCHE ET MÉCHANT 4 - 4,38 millions d'entrées
\end{enumerate}

\textit{Démarche :} Onglet 2024, appliquer un AutoFiltre, filtrer la nationalité (ETATS UNIS uniquement), filtrer la date de sortie (décocher 2023), trier par entrées décroissantes, utiliser \texttt{SOUS.TOTAL(3;...)} pour compter.

\vspace{0.5cm}
\textbf{Question 9 :} Quel est le total des entrées des films américains sortis en 2024 (en millions) ?

\textcolor{blue}{\textbf{Réponse :}} Le total est d'environ \textbf{40,51 millions d'entrées}.

\textit{Démarche :} Avec les filtres de la question précédente, utiliser \texttt{=SOUS.TOTAL(9;E8:E43)}.

\vspace{0.5cm}
\textbf{Question 10 :} Combien de films britanniques (GRANDE BRETAGNE) ont réalisé plus d'un million d'entrées en 2024 ?

\textcolor{blue}{\textbf{Réponse :}} \textbf{6 films britanniques} ont réalisé plus d'un million d'entrées en 2024.

\textit{Démarche :} Onglet 2024, filtrer sur nationalité = GRANDE BRETAGNE, compter avec \texttt{SOUS.TOTAL(3;...)}.

\vspace{0.5cm}
\textbf{Question 11 :} Quel est le film français ayant réalisé le moins d'entrées (mais plus d'un million) en 2024 ? Donnez son titre et son nombre d'entrées.

\textcolor{blue}{\textbf{Réponse :}} Le film français avec le moins d'entrées est \textbf{\og{}EMILIA PEREZ\fg{}} avec 1,08 million d'entrées (1~083~748 entrées).

\textit{Démarche :} Onglet 2024, filtrer sur FRANCE, trier la colonne entrées par ordre croissant, regarder la dernière ligne filtrée.

\subsection*{Questions sur l'année 2023}

\textbf{Question 12 :} Placez-vous dans l'onglet 2023. Combien de films français ont réalisé plus d'un million d'entrées en 2023 ?

\textcolor{blue}{\textbf{Réponse :}} \textbf{15 films français} ont réalisé plus d'un million d'entrées en 2023.

\textit{Démarche :} Onglet 2023, AutoFiltre, filtrer sur FRANCE, utiliser \texttt{=SOUS.TOTAL(3;B8:B50)}.

\vspace{0.5cm}
\textbf{Question 13 :} Quel est le film américain ayant fait le plus d'entrées en 2023 ? Combien d'entrées a-t-il réalisées (en millions) ?

\textcolor{blue}{\textbf{Réponse :}} Le film américain ayant réalisé le plus d'entrées en 2023 est \textbf{\og{}SUPER MARIO BROS. LE FILM\fg{}} avec \textbf{7,25 millions d'entrées} (7~249~542 entrées).

\textit{Démarche :} Onglet 2023, trier par entrées décroissantes, identifier le premier film américain (ou filtrer sur US et regarder le rang 1).

\vspace{0.5cm}
\textbf{Question 14 :} En 2023, un film japonais (JP) a réalisé plus d'un million d'entrées. Quel est son titre ?

\textcolor{blue}{\textbf{Réponse :}} Il s'agit du film \textbf{\og{}LE GARÇON ET LE HÉRON\fg{}} avec 1,55 million d'entrées.

\textit{Démarche :} Onglet 2023, filtrer sur nationalité = JP.

\vspace{0.5cm}
\textbf{Question 15 :} Certains films de 2023 sont sortis en 2022 (comme \textit{Avatar : La voie de l'eau}). Combien de films sortis en 2022 apparaissent dans la liste 2023 ?

\textcolor{blue}{\textbf{Réponse :}} \textbf{2 films} sortis en 2022 apparaissent dans la liste 2023 :
\begin{itemize}
\item AVATAR : LA VOIE DE L'EAU (sorti le 14/12/2022)
\item LE CHAT POTTÉ 2 : LA DERNIÈRE QUÊTE (sorti le 07/12/2022)
\end{itemize}

\textit{Démarche :} Onglet 2023, trier la colonne sortie par date, observer les films de fin 2022.

\subsection*{Comparaison 2023-2024}

\textbf{Question 16 :} Comparez le nombre total de films ayant réalisé plus d'un million d'entrées en 2023 et en 2024. Quelle année a connu le plus de succès au cinéma ?

\textcolor{blue}{\textbf{Réponse :}}
\begin{itemize}
\item En 2023 : 43 films ont réalisé plus d'un million d'entrées
\item En 2024 : 36 films ont réalisé plus d'un million d'entrées
\end{itemize}

L'année \textbf{2023} a connu plus de succès en termes de nombre de films à succès. Cependant, en 2024, le film \og{}UN P'TIT TRUC EN PLUS\fg{} a réalisé un record avec 10,72 millions d'entrées, soit plus que le meilleur film de 2023 (7,25 millions).

\textit{Démarche :} Compter le nombre de lignes de données dans chaque onglet (ligne 7 à la dernière ligne avec données).

\vspace{0.5cm}
\textbf{Question 17 :} En utilisant les fonctions de tri et de filtre, identifiez quel film français a réalisé le plus d'entrées sur les deux années 2023 et 2024 confondues. Donnez son titre, son année et son nombre d'entrées.

\textcolor{blue}{\textbf{Réponse :}} Le film français ayant réalisé le plus d'entrées est \textbf{\og{}UN P'TIT TRUC EN PLUS\fg{}} sorti en \textbf{2024} avec \textbf{10,72 millions d'entrées}.

\textit{Démarche :} Comparer les films français des deux années. Dans l'onglet 2024, filtrer sur FRANCE et noter le meilleur (10,72M). Dans l'onglet 2023, filtrer sur FRANCE et noter le meilleur (4,61M). Comparer les deux.

\section*{Pour aller plus loin}

\textbf{Question 18 :} Créez un tableau croisé dynamique sur les données de 2024. En utilisant ce tableau, répondez aux questions suivantes :
\begin{itemize}
\item Quelle est la nationalité qui a le plus grand nombre de films ayant dépassé le million d'entrées ?
\item Quelle nationalité a généré le plus d'entrées au total (en millions) ?
\item Ces deux réponses sont-elles identiques ? Pourquoi ?
\end{itemize}

\textcolor{blue}{\textbf{Réponse :}}

\textit{Démarche :} Sélectionner toutes les données de l'onglet 2024, menu \textit{Insertion} → \textit{Tableau croisé dynamique}, glisser \og{}nationalité\fg{} dans \og{}Champs de lignes\fg{}, \og{}titre\fg{} et \og{}entrées\fg{} dans \og{}Champs de données\fg{}.

Le tableau croisé dynamique produit les résultats suivants pour 2024 :

\begin{tabular}{|l|c|c|}
\hline
\textbf{Nationalité} & \textbf{Nombre de films} & \textbf{Total entrées (millions)} \\
\hline
ETATS UNIS & 17 & 44,05 \\
FRANCE & 9 & 34,41 \\
GRANDE BRETAGNE & 8 & 15,53 \\
FRANCE / CANADA & 2 & 2,13 \\
\hline
\end{tabular}

\vspace{0.5cm}
\textbf{Réponses aux trois questions :}
\begin{itemize}
\item \textbf{Nationalité avec le plus de films :} ETATS UNIS avec 17 films
\item \textbf{Nationalité avec le plus d'entrées :} ETATS UNIS avec 44,05 millions d'entrées
\item \textbf{Les réponses sont-elles identiques ?} OUI, les deux réponses désignent les ETATS UNIS. Cela s'explique par le fait que les films américains sont à la fois nombreux (17 films) et très populaires. Même si la France a des films à très gros succès (comme \og{}UN P'TIT TRUC EN PLUS\fg{} avec 10,72M), les États-Unis dominent par le volume total grâce à leur grand nombre de productions (presque le double de la France).
\end{itemize}

\vspace{0.5cm}
\textbf{Question 19 (Défi) :} Créez un nouveau tableau croisé dynamique pour analyser les sorties de films en 2024 selon les périodes de l'année.

\textcolor{blue}{\textbf{Réponse :}}

\textit{Démarche :} Créer un nouveau tableau croisé dynamique en glissant la colonne \og{}sortie\fg{} dans \og{}Champs de lignes\fg{}. LibreOffice peut grouper automatiquement par mois. Ajouter \og{}titre\fg{} (pour compter) et \og{}entrées\fg{} (pour sommer) dans \og{}Champs de données\fg{}. On peut ensuite regrouper les mois par trimestre manuellement ou analyser directement par mois.

\textbf{Analyse par trimestre (janvier-mars = T1, avril-juin = T2, etc.) :}

\begin{tabular}{|l|c|c|}
\hline
\textbf{Trimestre} & \textbf{Nombre de films} & \textbf{Total entrées (millions)} \\
\hline
T1 (janv-mars) & 8 & 15,67 \\
T2 (avr-juin) & 8 & 35,60 \\
T3 (juil-sept) & 6 & 13,25 \\
T4 (oct-déc) & 14 & 31,60 \\
\hline
\end{tabular}

\vspace{0.5cm}
\textbf{Réponses détaillées :}
\begin{itemize}
\item \textbf{Trimestre avec le plus de sorties :} T4 (octobre-décembre) avec 14 films. C'est la période des fêtes de fin d'année, traditionnellement forte pour le cinéma.

\item \textbf{Trimestre avec le plus d'entrées :} T2 (avril-juin) avec 35,60 millions d'entrées, malgré seulement 8 sorties. Ce trimestre inclut des blockbusters majeurs comme \og{}LE COMTE DE MONTE-CRISTO\fg{} (juin, 9,39M) et \og{}UN P'TIT TRUC EN PLUS\fg{} (mai, 10,72M).

\item \textbf{Corrélation ?} NON, il n'y a PAS de corrélation directe. Le trimestre avec le plus de sorties (T4, 14 films) n'est PAS celui avec le plus d'entrées (T2, 35,6M). Cela démontre qu'un petit nombre de très gros succès peut générer plus d'entrées que beaucoup de films moyens. La qualité et l'attractivité des films comptent plus que leur nombre.
\end{itemize}

\textit{Analyse complémentaire par mois :}
\begin{itemize}
\item Février a le plus de sorties (7 films)
\item Juin a généré le plus d'entrées (18,92 millions) grâce notamment au \og{}COMTE DE MONTE-CRISTO\fg{}
\end{itemize}

\vspace{0.5cm}
\noindent \fbox{\begin{minipage}{0.95\textwidth}
\textbf{Limite importante de cette analyse (esprit critique) :}

Notre analyse regroupe les entrées par \textbf{date de sortie}, pas par \textbf{période réelle de fréquentation}.

\textbf{Exemple concret :} \og{}LE COMTE DE MONTE-CRISTO\fg{} est sorti le 28 juin 2024 et a fait 9,39 millions d'entrées. Mais ces entrées ont été réalisées sur plusieurs mois (juin, juillet, août, septembre...). Dans notre analyse, tout est attribué à \og{}juin\fg{} ou au \og{}T2\fg{}.

\textbf{Ce que mesure vraiment notre analyse :}
\begin{itemize}
\item Quelles périodes de l'année voient sortir les films qui auront le plus de succès
\item Les stratégies de sortie des distributeurs (ils choisissent mai-juin pour les blockbusters)
\end{itemize}

\textbf{Ce que nous ne mesurons PAS :}
\begin{itemize}
\item Quand les spectateurs vont réellement au cinéma (il faudrait des données hebdomadaires d'entrées)
\item La répartition réelle de la fréquentation sur l'année
\end{itemize}

\textbf{Conclusion :} Cette analyse reste pertinente pour comprendre les stratégies de l'industrie cinématographique, mais elle ne reflète pas directement les périodes de forte fréquentation des salles. Pour cela, il faudrait des données plus détaillées sur la répartition temporelle des entrées de chaque film.
\end{minipage}}

\vspace{1cm}
\section*{Rappel des principales fonctions SOUS.TOTAL}

La fonction \texttt{SOUS.TOTAL} permet d'effectuer des calculs uniquement sur les données filtrées :
\begin{itemize}
\item \texttt{SOUS.TOTAL(1;plage)} : Moyenne
\item \texttt{SOUS.TOTAL(3;plage)} : Nombre de cellules non vides (NBVAL)
\item \texttt{SOUS.TOTAL(9;plage)} : Somme
\end{itemize}

\vspace{1cm}
\noindent \textbf{Fin du corrigé}

\end{document}
