\documentclass[12pt,a4paper]{article}
\usepackage[utf8]{inputenc}
\usepackage[french]{babel}
\usepackage[T1]{fontenc}
\usepackage{geometry}
\usepackage{hyperref}
\usepackage{enumitem}
\usepackage{xcolor}
\usepackage{listings}
\usepackage{textcomp}
\usepackage{pifont}

\geometry{margin=2.5cm}

% Définition du symbole crayon
\newcommand{\crayon}{\ding{46}}

% Configuration pour le code Python
\lstset{
  language=Python,
  basicstyle=\ttfamily\small,
  keywordstyle=\color{blue},
  commentstyle=\color{gray},
  stringstyle=\color{red},
  showstringspaces=false,
  breaklines=true,
  frame=single,
  numbers=left,
  numberstyle=\tiny\color{gray},
  extendedchars=true,
  inputencoding=utf8,
  mathescape=false,
  upquote=true,
  columns=flexible,
  keepspaces=true,
  literate=
    {à}{{\`a}}1 {â}{{\^a}}1 {é}{{\'e}}1 {è}{{\`e}}1 {ê}{{\^e}}1
    {î}{{\^i}}1 {ô}{{\^o}}1 {ù}{{\`u}}1 {û}{{\^u}}1
    {ë}{{\"e}}1 {ï}{{\"i}}1 {ü}{{\"u}}1 {ç}{{\c{c}}}1
}

\title{\textbf{Guide des commandes Pandas}\\
\large Aide-mémoire pour le TP sur les données structurées}
\author{}
\date{}

\begin{document}

\maketitle

\section*{Introduction}

Ce document accompagne le notebook Jupyter \texttt{analyse\_films\_capytale.ipynb} disponible sur Capytale. Il présente les principales commandes Pandas que vous utiliserez pendant le TP.

\section{Qu'est-ce que Pandas ?}

\textbf{Pandas} est une bibliothèque Python très puissante pour l'analyse de données. Elle permet de manipuler des tableaux de données (appelés \textit{DataFrames}) de manière simple et efficace.

Un \textbf{DataFrame} est similaire à un tableau Excel : il a des lignes, des colonnes avec des noms, et on peut filtrer, trier et calculer des statistiques dessus.

\section{Commandes de base}

\subsection{Afficher les données}

\begin{lstlisting}
# Afficher les 5 premières lignes
df.head()

# Afficher les 10 premières lignes
df.head(10)

# Afficher les 5 dernières lignes
df.tail()
\end{lstlisting}

\subsection{Informations sur le DataFrame}

\begin{lstlisting}
# Afficher les noms des colonnes
df.columns
# ou
df.columns.tolist()

# Afficher des informations générales (types, mémoire)
df.info()

# Afficher les dimensions (nombre de lignes, nombre de colonnes)
df.shape
# Exemple de résultat : (36, 5) signifie 36 lignes et 5 colonnes

# Afficher les statistiques descriptives
df.describe()
\end{lstlisting}

\subsection{Sélectionner une colonne}

\begin{lstlisting}
# Sélectionner une seule colonne
df['nationalité']

# Sélectionner plusieurs colonnes
df[['titre', 'entrées (millions)']]
\end{lstlisting}

\subsection{Sélectionner une ligne}

\begin{lstlisting}
# Sélectionner la première ligne (index 0)
df.iloc[0]

# Sélectionner les 3 premières lignes
df.iloc[0:3]

# Sélectionner la dernière ligne
df.iloc[-1]
\end{lstlisting}

\section{Filtrer les données}

Le filtrage est l'une des opérations les plus importantes. On utilise des \textbf{conditions} pour ne garder que certaines lignes.

\subsection{Filtrage simple}

\begin{lstlisting}
# Filtrer les films français
films_francais = df[df['nationalité'] == 'FRANCE']

# Filtrer les films américains
films_us = df[df['nationalité'] == 'ETATS UNIS']

# Filtrer les films avec plus de 5 millions d'entrées
gros_succes = df[df['entrées (millions)'] > 5]
\end{lstlisting}

\subsection{Filtrage avec plusieurs conditions}

\begin{lstlisting}
# ET logique : &
# Films français avec plus de 2 millions d'entrées
films_fr_succes = df[(df['nationalité'] == 'FRANCE') &
                      (df['entrées (millions)'] > 2)]

# OU logique : |
# Films français OU britanniques
films_fr_ou_gb = df[(df['nationalité'] == 'FRANCE') |
                     (df['nationalité'] == 'GRANDE BRETAGNE')]
\end{lstlisting}

\textbf{Attention :} N'oubliez pas les parenthèses autour de chaque condition !

\subsection{Filtrage sur les dates}

\begin{lstlisting}
# D'abord, convertir la colonne en format date
df['sortie'] = pd.to_datetime(df['sortie'])

# Extraire l'année
df['année_sortie'] = df['sortie'].dt.year

# Filtrer sur l'année
films_2024 = df[df['année_sortie'] == 2024]

# Extraire le mois
df['mois'] = df['sortie'].dt.month
\end{lstlisting}

\section{Trier les données}

\begin{lstlisting}
# Trier par ordre croissant (du plus petit au plus grand)
df_trie = df.sort_values('entrées (millions)')

# Trier par ordre décroissant (du plus grand au plus petit)
df_trie = df.sort_values('entrées (millions)', ascending=False)

# Trier sur plusieurs colonnes
df_trie = df.sort_values(['nationalité', 'entrées (millions)'],
                          ascending=[True, False])
\end{lstlisting}

\section{Compter et calculer}

\subsection{Compter le nombre de lignes}

\begin{lstlisting}
# Compter le nombre de films
nombre = len(df)

# Autre méthode
nombre = df.shape[0]

# Compter les valeurs non nulles dans une colonne
nombre = df['titre'].count()
\end{lstlisting}

\subsection{Calculer des statistiques}

\begin{lstlisting}
# Calculer la somme d'une colonne
total = df['entrées (millions)'].sum()

# Calculer la moyenne
moyenne = df['entrées (millions)'].mean()

# Calculer le minimum
minimum = df['entrées (millions)'].min()

# Calculer le maximum
maximum = df['entrées (millions)'].max()

# Résumé statistique complet
df['entrées (millions)'].describe()
\end{lstlisting}

\section{Grouper et agréger}

La méthode \texttt{groupby()} permet de regrouper les données selon une catégorie et de calculer des statistiques pour chaque groupe.

\begin{lstlisting}
# Grouper par nationalité et compter les films
stats = df.groupby('nationalité')['titre'].count()

# Grouper par nationalité et sommer les entrées
stats = df.groupby('nationalité')['entrées (millions)'].sum()

# Grouper et calculer plusieurs statistiques à la fois
stats = df.groupby('nationalité').agg({
    'titre': 'count',              # Compter les films
    'entrées (millions)': 'sum'    # Sommer les entrées
})

# Renommer les colonnes du résultat
stats.columns = ['Nombre de films', 'Total entrées']

# Trier les résultats
stats = stats.sort_values('Total entrées', ascending=False)
\end{lstlisting}

\section{Enchaîner les opérations}

On peut enchaîner plusieurs opérations à la suite :

\begin{lstlisting}
# Filtrer les films français, trier par entrées décroissantes,
# et afficher les 3 premiers
top3_fr = df[df['nationalité'] == 'FRANCE'] \
    .sort_values('entrées (millions)', ascending=False) \
    .head(3)

# Afficher seulement certaines colonnes
top3_fr = df[df['nationalité'] == 'FRANCE'] \
    .sort_values('entrées (millions)', ascending=False) \
    .head(3)[['titre', 'entrées (millions)']]
\end{lstlisting}

\section{Astuces et bonnes pratiques}

\subsection{Affichage amélioré}

\begin{lstlisting}
# Afficher plus de lignes
pd.set_option('display.max_rows', 100)

# Afficher plus de colonnes
pd.set_option('display.max_columns', 20)

# Ne pas tronquer les noms de colonnes
pd.set_option('display.max_colwidth', None)
\end{lstlisting}

\subsection{Vérifier les valeurs uniques}

\begin{lstlisting}
# Afficher toutes les nationalités présentes
df['nationalité'].unique()

# Compter le nombre de valeurs uniques
df['nationalité'].nunique()

# Compter les occurrences de chaque valeur
df['nationalité'].value_counts()
\end{lstlisting}

\subsection{Gérer les valeurs manquantes}

\begin{lstlisting}
# Vérifier s'il y a des valeurs manquantes
df.isnull().sum()

# Supprimer les lignes avec des valeurs manquantes
df_clean = df.dropna()

# Remplacer les valeurs manquantes
df['colonne'].fillna(0)
\end{lstlisting}

\section{Tableau récapitulatif}

\begin{center}
\begin{tabular}{|p{5cm}|p{8cm}|}
\hline
\textbf{Opération} & \textbf{Commande} \\
\hline
Afficher les premières lignes & \texttt{df.head()} \\
\hline
Afficher les colonnes & \texttt{df.columns} \\
\hline
Sélectionner une colonne & \texttt{df['nom\_colonne']} \\
\hline
Filtrer & \texttt{df[df['colonne'] == valeur]} \\
\hline
Trier (décroissant) & \texttt{df.sort\_values('colonne', ascending=False)} \\
\hline
Compter les lignes & \texttt{len(df)} ou \texttt{df.shape[0]} \\
\hline
Calculer une somme & \texttt{df['colonne'].sum()} \\
\hline
Calculer une moyenne & \texttt{df['colonne'].mean()} \\
\hline
Grouper et agréger & \texttt{df.groupby('cat').agg(...)} \\
\hline
Première ligne & \texttt{df.iloc[0]} \\
\hline
Statistiques descriptives & \texttt{df.describe()} \\
\hline
\end{tabular}
\end{center}

\section*{Ressources complémentaires}

\begin{itemize}
\item Documentation officielle Pandas : \url{https://pandas.pydata.org/docs/}
\item Tutoriel Pandas (français) : \url{https://openclassrooms.com/fr/courses/4452741-decouvrez-les-librairies-python-pour-la-data-science}
\end{itemize}

\vspace{1cm}
\noindent \textbf{Bon travail avec le notebook Jupyter sur Capytale !}

\end{document}
