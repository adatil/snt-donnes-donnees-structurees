\documentclass[12pt,a4paper]{article}
\usepackage[utf8]{inputenc}
\usepackage[french]{babel}
\usepackage[T1]{fontenc}
\usepackage{geometry}
\usepackage{hyperref}
\usepackage{enumitem}
\usepackage{xcolor}
\usepackage{pifont}

\geometry{margin=2.5cm}

% Définition du symbole crayon
\newcommand{\crayon}{\ding{46}}

\title{\textbf{TP sur les données structurées}\\
\large Partie 1 : Tableur et tris\\
Films ayant réalisé plus d'un million d'entrées}
\author{}
\date{}

\begin{document}

\maketitle

\section*{Préparation du compte-rendu}

\begin{enumerate}
\item Ouvrez le traitement de texte LibreOffice Writer
\item Écrivez vos noms, la date, le titre de la séance
\item Enregistrez le document dans votre répertoire personnel, en lui donnant pour nom \texttt{snt-chap4-tableur-NOMS} (en remplaçant NOMS par vos noms de famille)
\item Convertissez le document en PDF (pour vérifier que vous savez comment faire). Vous ferez cela à nouveau à la fin de l'activité, au moment de rendre le compte-rendu.
\end{enumerate}

\vspace{0.5cm}
\textbf{NB :} \textcolor{blue}{\crayon} : Ce symbole indique que vous devez répondre à la question dans votre compte-rendu

\section{Téléchargement et Ouverture du fichier}

\begin{enumerate}[start=0]
\item Lancez un navigateur web et rendez-vous sur la plateforme ouverte des données publiques françaises : \url{https://www.data.gouv.fr/fr/}
\item Recherchez \textbf{Films entrées} dans le champ de recherche
\item Téléchargez le fichier \textit{Films ayant réalisé plus d'un million d'entrées} au format \texttt{.xlsx}
\item Ouvrez le logiciel LibreOffice Calc, et ouvrez le fichier que vous venez de télécharger
\end{enumerate}

\section{Analyse sommaire du fichier}

\textcolor{blue}{\crayon} \textbf{Question 1 :} Quel est le film ayant fait le plus d'entrées en 2024 ?

\textcolor{blue}{\crayon} \textbf{Question 2 :} Quel est le film français ayant fait le plus d'entrées en 2023 ?

\textcolor{blue}{\crayon} \textbf{Question 3 :} Quelle est la nationalité du film ayant fait le plus d'entrées en 2023 ?

\section{Utilisation des filtres}

\begin{enumerate}
\item Placez-vous dans l'onglet de l'année \textbf{2024}
\item Cliquez sur la gauche de la ligne 7 (pour la sélectionner entièrement)
\item Dans le menu, cliquez sur \textit{Données} puis \textit{AutoFiltre}
\item Filtrez par nationalité, décochez tous les éléments sauf \texttt{FRANCE}, puis validez avec le bouton OK
\item Vous ne devriez voir afficher que les films français de l'année 2024
\end{enumerate}

\textcolor{blue}{\crayon} \textbf{Question 4 :} Quels sont les trois films français ayant fait le plus d'entrées en 2024 ?

\vspace{0.5cm}
\noindent On souhaite connaître le nombre de films français de cette liste. Dans une cellule vide (par exemple B51), écrivez \texttt{=SOUS.TOTAL(3;B8:B49)} (cette fonction permet de calculer le nombre de cellules de la colonne B, mais seulement pour les cellules filtrées).

\vspace{0.3cm}
\noindent On souhaite connaître le nombre d'entrées réalisées par les films français de cette liste. Dans la cellule E51, écrivez \texttt{=SOUS.TOTAL(9;E8:E49)} (cette fonction permet de calculer la somme de la colonne E, mais seulement pour les films filtrés).

\vspace{0.5cm}
\textcolor{blue}{\crayon} \textbf{Question 5 :} Combien de films français ont réalisé plus d'un million d'entrées en 2024 ?

\textcolor{blue}{\crayon} \textbf{Question 6 :} Quel est le total d'entrées réalisées par ces films français en 2024 (en millions) ?

\vspace{0.5cm}
\noindent Supprimez le filtre sur la nationalité (vous devez à nouveau voir toutes les nationalités), puis triez la colonne \textit{sortie} par date.

\textcolor{blue}{\crayon} \textbf{Question 7 :} Parmi les films affichés en 2024, certains sont sortis en 2023 (comme \textit{Wonka} sorti le 13 décembre 2023). Expliquez pourquoi ces films apparaissent dans la liste des films les plus vus en 2024.

\section{Exercice : Analyse des films de 2023 et 2024}

Pour toutes les questions suivantes, précisez la démarche utilisée : Quels filtres avez-vous utilisés ? Quels tris ? Quelles fonctions (dans quelles cellules) ? Quelles analyses ou manipulations avez-vous effectuées \og{}à la main\fg{} le cas échéant ?

\subsection*{Questions sur l'année 2024}

\textcolor{blue}{\crayon} \textbf{Question 8 :} Combien de films américains (ETATS UNIS) sortis en 2024 (et non en 2023) ont réalisé plus d'un million d'entrées ? Donnez également le titre des trois premiers.

\textcolor{blue}{\crayon} \textbf{Question 9 :} Quel est le total des entrées des films américains sortis en 2024 (en millions) ?

\textcolor{blue}{\crayon} \textbf{Question 10 :} Combien de films britanniques (GRANDE BRETAGNE) ont réalisé plus d'un million d'entrées en 2024 ?

\textcolor{blue}{\crayon} \textbf{Question 11 :} Quel est le film français ayant réalisé le moins d'entrées (mais plus d'un million) en 2024 ? Donnez son titre et son nombre d'entrées.

\subsection*{Questions sur l'année 2023}

\textcolor{blue}{\crayon} \textbf{Question 12 :} Placez-vous dans l'onglet 2023. Combien de films français ont réalisé plus d'un million d'entrées en 2023 ?

\textcolor{blue}{\crayon} \textbf{Question 13 :} Quel est le film américain ayant fait le plus d'entrées en 2023 ? Combien d'entrées a-t-il réalisées (en millions) ?

\textcolor{blue}{\crayon} \textbf{Question 14 :} En 2023, un film japonais (JP) a réalisé plus d'un million d'entrées. Quel est son titre ?

\textcolor{blue}{\crayon} \textbf{Question 15 :} Certains films de 2023 sont sortis en 2022 (comme \textit{Avatar : La voie de l'eau}). Combien de films sortis en 2022 apparaissent dans la liste 2023 ?

\subsection*{Comparaison 2023-2024}

\textcolor{blue}{\crayon} \textbf{Question 16 :} Comparez le nombre total de films ayant réalisé plus d'un million d'entrées en 2023 et en 2024. Quelle année a connu le plus de succès au cinéma ?

\textcolor{blue}{\crayon} \textbf{Question 17 :} En utilisant les fonctions de tri et de filtre, identifiez quel film français a réalisé le plus d'entrées sur les deux années 2023 et 2024 confondues. Donnez son titre, son année et son nombre d'entrées.

\section*{Pour aller plus loin (optionnel)}

\textcolor{blue}{\crayon} \textbf{Question 18 :} Téléchargez le fichier \textit{Liste des établissements cinématographiques actifs} sur data.gouv.fr. Combien de cinémas \og{}Art et Essai\fg{} sont équipés d'un écran 3D ?

\vspace{1cm}
\noindent \textbf{Fin du TP - N'oubliez pas de convertir votre compte-rendu en PDF et de le déposer sur Moodle !}

\end{document}
